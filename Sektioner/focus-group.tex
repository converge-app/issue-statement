\chapter{Målgruppe}

Det egentlige produkt er tilegnet, folk med kompetencer der kan gøre sig gældende som kontrakt arbejde, som web-design, grafisk-design, oversættelse osv. Desuden kan det også være tilgængeligt for entreprenører og selvstændige der leder efter arbejdskraft.

Samt folk der er klar på nye udfordringer og oplevelser, samt at kunne tage projektarbejde. Derfor vil målgruppen være delt i to, et for udbydere og et for freelancere (Employers, Freelancers).

Udbydere er målrettet mod små til mellemstore virksomheder som har brug for kort varigt konsulent arbejde. F.eks. at kode en hjemmeside, eller lave et design til et af deres produkter. Dette vil typisk enten være håndteret af en salgsrepræsentant for det firma, eller ved en meget lille virksomhed, for et tømmerfirma, kunne det være ejeren selv, hvilket er en læperson på området. Men kunne også være en selvstændig murermester, som har brug for et nyt print til sin varevogn.

Freelancere er primært målrettet mod selvstændige der gerne vil have muligheden for at arbejde med projekter rundt om på kloden. Dette er typisk personer der er villige til at arbejde mange timer, til at nå en hård deadline, i bytte for ekstra betaling.

\section{Projektets anvendelsesområde}

Meningen med Converge er at modernisere og effektivisere arbejdskraft omkring på kloden. Til dette skal der fremstilles et web-interface der tillader adgang på adskillelige platforme. Dernæst skal der laves et fuldstændigt backend til at understøtte samarbejdedt mellem de forskellige brugere og deres transaktioner.

Til at understøtte de mange forskellige krav, som f.eks. betaling, kommunikation, samarbejde med mere. Bruges der et cloud-native mindset til at gøre det muligt at have en agil instilling til software udvikling. Dette vil munde ud i en decentraliseret microservices platform, hvor hver service opfylder et specifikt krav for produktet. Om det er til at håndtere logins eller til at føre betalinger.

Alt dette er med til at gøre det muligt for både freelancere og købere at opnå deres mål med henholdsvis at tjene penge, og få lavet et stykke arbejde efter ønske.

\section{Persona}

Til Converge er der oprettet et antal personaner, disse personaer er beskrevet i dokumentet \cite{document-requirements}. Og skal hjælpe med at beskrive de forskellige krav i forhold til problemstillingen.

\section{Test personer}

Til Converge er der valgt at brug en lille gruppe af test personer, som kan sættes sig i rollen som rigtige aktører og sætte fokus på de elementer som brugeren føler er mest vigtige.